%!TEX TS-program = uplatex
%!TEX encoding = UTF-8 Unicode
%!TEX spellcheck = ja-JP
\documentclass[autodetect-engine]{jsarticle}

\usepackage{wtref}

\newref[namespace=foo, scope=section]{seq, tab}
\newref{eq, sec}

% error cases
%\newref{eq}                   % 再定義
%\newref[scope=section!]{bar}  % 存在しないカウンタ

\setrefstyle{eq, seq}{prefix=式, refcmd=(\ref{#1})}
\setrefstyle{sec}{prefix=第, suffix=節}
\setrefstyle{tab}{prefix=表, refcmd=\ref{#1} (p.~\pageref{#1})}

\title{ピタゴラスの定理について}
\author{ワトソン}

\begin{document}

\maketitle

\section{ピタゴラスの定理}\seclabel{ピタゴラスの定理}

\tabref{生涯}のような生涯を過ごしたピタゴラスは直角三角形の各辺の長さについて
%
\begin{equation}
a^2 + b^2 = c^2 \eqlabel{ピタゴラスの定理1}
\end{equation}
%
という関係が成り立つこと(ピタゴラスの定理)を発見したとされる.上記の関係は
%
\begin{equation}
a^2 + b^2 - c^2 = 0 \eqlabel{ピタゴラスの定理2}
\end{equation}
%
と表すこともできる.\eqref{ピタゴラスの定理1, ピタゴラスの定理2}は等価である.


\begin{table}[h]
\centering
\caption{ピタゴラスの生涯}\tablabel{生涯}
\begin{tabular}{ll}
西暦 & 出来事 \\ \hline
紀元前582年 & 生まれる \\
紀元前496年 & 生涯を終える \\
\end{tabular}
\end{table}

\section{余弦定理}

ピタゴラスの定理を一般化したものが余弦定理である.
\begin{equation}
a^2=b^2+c^2-2bc\cos A \seqlabel{余弦定理}
\end{equation}
この\seqref{余弦定理}が余弦定理である.

\section{復習}

\secref{ピタゴラスの定理}で述べたように,ピタグラスの定理は\eqref{ピタゴラスの定理1}で
表され,余弦定理は\seqref{余弦定理}で表される.表\tabref{生涯}によれば,この定理は
紀元前587年から紀元前496年の間に発見されたことになるが,それ以前から知られていたという
説もある.

\end{document}
