\documentclass[a4paper]{article}

\usepackage{enumitem}
\usepackage{shortvrb}
\MakeShortVerb{\|}
\newcommand{\PkgName}{\textsf{WTRef}}
\newcommand{\Meta}[1]{$\langle$\mbox{}\textit{#1}\mbox{}$\rangle$}
\newenvironment{syntax}{\begin{quote}\small}{\end{quote}}

\title{{\PkgName} Package (v0.1)}
\author{Watson}

\begin{document}

\maketitle

\begin{abstract}
WT series collects macros which author frequently use to create {\LaTeX} documents.
{\PkgName} package is a part of this WT Series that extend {\LaTeX} original
cross-reference system. It make enable to divide namespace and arrow users to customise
reference format. {\LaTeXe} on any kind of {\TeX} engine is supported, but \textsf{xkeyval}
package is required.
\end{abstract}

\section{Loading the {\PkgName} Package}

To use {\PkgName} package, load \texttt{wtref.sty} file with |\usepackage| command in preamble.
On this occasion, you can specify the scope of closs-reference commands as package option.
%
\begin{syntax}
|\usepackage[|\Meta{scope}|]{wtref}|
\end{syntax}

You can choose from following four types for \Meta{scope}:
%
\begin{description}
\item[|chapter|]
Set the scope chapter-by-chapter. You can not use this option if |\chapter| is undefined
(an error will be raised).

\item[|section|]
Set the scope section-by-section.

\item[|subsection|]
Set the scope subsection-by-subsection.

\item[|global|]
Do not set any scope (default).
\end{description}

When you omit package option, \Meta{scope} will set to |global|.

\section{Cross-Reference Commands}

\subsection{Definition of New Cross-Reference Commands}

|\newref| command create a set of cross-reference commands.
%
\begin{syntax}
|\newref{|\Meta{ref type}|}|
\end{syntax}
%
This |\newref| command can only be used in preamble. In addition, all characters of
\Meta{ref type} must be able to use in control sequence (only ordinary alphabet is recommended)
and can not be empty.

|\newref| command defines two commands: |\|\Meta{ref type}|label|, |\|\Meta{ref type}|ref|.
In this documet, the formar are called \textbf{label commands} and the latter are called
\textbf{reference commands}. |\newref| command overwrites existing commands, so \Meta{ref name}
should be decided carefully.

\subsubsection{Internal Processing}

Label commands finaly are expanded to following format:
%
\begin{syntax}
|\label{|\Meta{ref type}|:|\Meta{scope num}|:|\Meta{label}|}|
\end{syntax}

A \Meta{scope num} is string which composed by arabic numbers and periods.
The format of \Meta{scope num} depends on \Meta{scope}:
%
\begin{description}
\item[|chapter|]
``\Meta{chapter num}''

\item[|section|]
``\Meta{chapter num}.\Meta{section num}'' or ``\Meta{section num}''

\item[|subsection|]
``\Meta{chapter num}.\Meta{section num}.\Meta{subsection num}'' or \\
``\Meta{section num}.\Meta{subsection num}''
\end{description}

However, if \Meta{scope} is set to |global|, label command will be expanded to following form:
%
\begin{syntax}
|\label{|\Meta{ref type}|:|\Meta{label}|}|
\end{syntax}

\subsection{Reference Commands}

Reference commands print contents of counters which labeled by label commands in specified
formats. Usage of |\exref| is shown bellow as an example:
%
\begin{syntax}
|\exref[|\Meta{scope num}|]{|\Meta{label list}|}|
\end{syntax}

The option argument \Meta{scope num} can be ommited when refering label exists in the
same scope. Especially, if \Meta{scope} is set to |global|, this argument is always
unnecessary, and in other words it will be ignored all the time.

In argument \Meta{label list}, plural labels can be written in comma-separated. If
actually plural labels are filled in, pertinent counters should be printed out in
comma-separated form in default. You can change this format flexiby with |\setrefstyle|
command.

\section{Setting Referece Style}

The output format of reference commands can be customised with |\setrefstyle| command.
The syntax of |\setrefstyle| is shown bellow:
%
\begin{syntax}
|\setrefstyle{|\Meta{ref type}|}{|\Meta{options}|}|
\end{syntax}

The |\setrefstyle| command can be used any place of {\LaTeX} document (not only preamble),
and change reference format locally.

In \Meta{options}, you can set following parameters by key-value list:
%
\begin{description}[font=\normalfont]
\item[|refcmd=|\Meta{command}]
Specified \Meta{command} repeated for the number of labels which filled in \Meta{label list}
time. String |#1| in \Meta{command} may be replaced into appropriate label name. The default
value is |\ref{#1}|.
%
\item[|sep=|\Meta{command}]
Specified \Meta{command} is output as a separater of each |refcmd| when more than three
labels filled in \Meta{label list}. Notice that last one separater is given by |last sep|.
The default value is |{,\space}|.
%
\item[|last sep|(|=|\Meta{command})]
Specified \Meta{command} is output as a last separater when plura labels filled in
\Meta{label list}. Behind the |=| can be ommited, and in that case |last sep| is set
to identical value of |sep| (and this is the default).
%
\item[|prefix=|\Meta{command}]
Specified \Meta{command} put out first when referece command used. The default value is |{}|.
%
\item[|suffix=|\Meta{command}]
Specified \Meta{command} put out last when referece command used. The default value is |{}|.
\end{description}

Parameters which do not set explicitly will not be changed.

\end{document}
