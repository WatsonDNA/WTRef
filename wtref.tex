%#!pdflatex
\documentclass[a4paper]{article}

\usepackage{enumitem}
\usepackage{shortvrb}

\MakeShortVerb{\|}
\newcommand{\PkgName}[1]{\textsf{#1}}
\newcommand{\ThisPkg}{\PkgName{WTRef}}
\newcommand{\Meta}[1]{$\langle$\mbox{}\textit{#1}\mbox{}$\rangle$}
\newenvironment{syntax}{\begin{quote}\small}{\end{quote}}
\newcommand{\eg}{e.g.,\,}
\newcommand{\ie}{i.e.,\,}

\title{The {\ThisPkg} Package (v0.4.0)}
\author{Takuto ASAKURA (wtsnjp)}

\begin{document}

\maketitle

\begin{abstract}
Package {\ThisPkg} is a small extension for {\LaTeXe}'s cross-referencing. It
enables you to divide namespace and scope. It also provides a few ways to
customise referencing formats. {\LaTeXe} on any kind of {\TeX} engine is
supported. The package requires Package \PkgName{xparse} and \PkgName{xkeyval}.
\end{abstract}

\section{System requirements}

{\ThisPkg} requires the following.
%
\begin{itemize}
\item {\TeX} engine: any engine
\item {\TeX} format: \LaTeXe
\item Document class: any class
\item Required packages: \PkgName{xparse} and \PkgName{xkeyval}
\end{itemize}

\section{Loading the package}

The package should be loaded in the usual {\LaTeXe} way. No package option is
available.
%
\begin{syntax}
|\usepackage{wtref}|
\end{syntax}

\section{Cross-reference commands}

\subsection{Declaring new cross-reference commands}

The |\newref| command creates a pair of cross-reference commands. This command
can be used in preamble only.
%
\begin{syntax}
|\newref[|\Meta{options}|]{|\Meta{ref types}|}|
\end{syntax}
%
Herein, \Meta{ref types} is a comma-separated list of \Meta{ref type}. All
characters consisting of \Meta{ref type} must be `letter' so that they can be
used for control sequences (recommended to use ASCII alphabets only) and may
not be empty. Notice that leading and trailing spaces and successive spaces
around commas are ignored.

The |\newref| command defines pairs of cross-referencing commands, |\|\Meta{ref
type}|label| and |\|\Meta{ref type}|ref| for each given \Meta{ref type}. In
this document, we call the former \emph{label commands} and the latter
\emph{reference commands}. Notice that the |\newref| command may overwrites
existing commands (with warning messages), so \Meta{ref name} should be decided
carefully.

In \Meta{options}, you can set the following parameters with key-value list:
%
\begin{description}[font=\normalfont]
\item[|namespace=|\Meta{string}] sets \Meta{namespace} to ``\Meta{string}:''.
In case neither |namespace| nor |nonamespace| is specified, or \Meta{string} of
|namespace| is empty, the \Meta{namespace} parts in labels are set to
``\Meta{ref type}:''.
%
\item[|nonamespace|] sets \Meta{namespace} to empty. That is to say, the
namespace function for the package is disabled. You can give a value for the
|nonamespace| key without any error, but the value will be simply ignored.
%
\item[|scope=|\Meta{counter}] sets counter which used as scope. Though you can
specify arbitrary {\LaTeX} counter for \Meta{counter}, normally those which
have uniqueness in a document are useful. The \Meta{scope} parts in labels are
set to ``|\the|\Meta{counter}:''.
\end{description}
%
These optional settings apply to all pairs of cross-reference commands
corresponding to \Meta{ref type}s in specified \Meta{ref types}.

Identically, if any keys do not specified in \Meta{options}, \Meta{namespace}
is set to ``\Meta{ref type}'', and \Meta{scope} is set to empty. In other words,
while the namespace function is enabled by default, the scope function is not.

\subsection{Label commands}

Label commands are used for making new labels. The usage of them are the same
as the standard |\label| command of {\LaTeXe}, \eg
%
\begin{syntax}
|\exlabel{|\Meta{label}|}|
\end{syntax}

The label commands are equivalent to the following after a full-expansion:
%
\begin{syntax}
|\label{|\Meta{namespace}\Meta{scope}\Meta{label}|}|
\end{syntax}

\subsection{Reference commands}

Reference commands print contents of counters which labeled by label commands
in specified formats. The following is the syntax of an example reference
command |\exref|:
%
\begin{syntax}
|\exref[|\Meta{the scope}|]{|\Meta{label list}|}|
\end{syntax}

The option argument \Meta{the scope} can be omitted when the specified label
exists in the same scope. You can refer to labels outside a scope by specifying
the target scope explicitly, that is the output of proper |\the|\Meta{counter},
in the option argument. Notice that if the function of scope is inactive (\ie
in case |scope| key does not specified in \Meta{options} of |\newref|), this
argument is always unnecessary, and it will be ignored all the time.

In argument \Meta{label list}, you can specify multiple labels in a form of
comma-separated list. Note that leading and trailing spaces and successive
spaces around commas are ignored. In the case, pertinent counters should be
printed in comma-separate form by default. You can change this format
flexibly with the |\setrefstyle| command.

\section{Setting referencing style}

The output format of reference commands can be customised with the
|\setrefstyle| command. The syntax of the command is as follows:
%
\begin{syntax}
|\setrefstyle{|\Meta{ref types}|}{|\Meta{options}|}|
\end{syntax}

The |\setrefstyle| command can be used anywhere in {\LaTeX} documents, not
limited to preambles, and sets the reference format locally.

In \Meta{options}, you can set the following parameters with a key-value list:
%
\begin{description}[font=\normalfont]
\item[|refcmd=|\Meta{ref command}] sets the \Meta{ref command} to used for actual
referencing functionality. Herein, |#1| in the \Meta{command} may be replaced
into suitable label name. The default value is |\ref{#1}|. The \Meta{command}
may be used repeatedly for the number of labels in the given \Meta{label list}.
%
\item[|sep=|\Meta{separator}] sets the \Meta{separator}, which used between
each \Meta{ref command}, in case more than three labels are given for a
\Meta{label list}. Notice that last one separator is set by |last sep|. The
default value is |{,\space}|.
%
\item[|last sep|(|=|\Meta{last separator})] sets \Meta{last separator} to used
for the last separator between the \Meta{ref command}s, in case multiple labels
are given in \Meta{label list}. The part after |=| can be omitted. In that case,
|last sep| is set to identical value of |sep|, and this is the default
behavior.
%
\item[|prefix=|\Meta{prefix}]
sets \Meta{prefix} to be put in front of the first \Meta{ref command} when the
reference commands are used. The default value is |{}|.
%
\item[|suffix=|\Meta{suffix}] sets \Meta{suffix} to be put behind the last
\Meta{ref command} when reference commands are used. The default value is |{}|.
\end{description}
%
Values of the keys which do not set explicitly in the \Meta{options} are left
unchanged.

\end{document}
% vim: spell:
